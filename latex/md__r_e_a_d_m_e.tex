Este es nuestro proyecto {\bfseries{RETINA (Versiones 1 y 2)}} del curso 2022-\/2023 de la asignatura de Sistemas Digitales II.

Este proyecto, con la implementación hardware que muestra la fotografía, consigue transmitir tramas NEC que muestran los colores ROJO, VERDE y AZUL en un receptor adecuado (este será implementado en la Versión 3 del mismo proyecto). Se adjunta una lista de elementos hardware para hacer esta prueba más abajo.

En el vídeo adjunto se pueden apreciar en el osciloscopio las tramas NEC enviadas y como cambia al respectivo color el LED RGB del receptor.

Puedes acceder al vídeo del demostrador del proyecto pinchando en la imagen\+:

\href{https://youtube.com/shorts/jffZE0cjdjA?feature=share}{\texttt{ }}

{\bfseries{Lista de materiales para la prueba\+:}}
\begin{DoxyItemize}
\item Nucleo-\/\+STM32\+F446\+RE -\/ 1 BASE L1, L2, L3
\item Cable USB. A Macho-\/B Mini Importante\+: cable de alimentación y datos 1 BASE Link
\item Latiguillos macho-\/macho (Cables de conexión para protoboard +10 BASE Link)
\item Protoboard Tamaño mínimo\+: 80x60 mm 1 BASE Link
\item Diodo emisor infrarrojos De inserción. (Encapsulado 5 mm. Longitud de onda central\+: ∼ 940 − 050 nm. Potencia max\+: 40 mW e.\+g.\+: 17◦ visión)
\item MOSFET puerta tipo N (Empaquetado\+: TO-\/92. Tensión max D-\/S\+: 40 − 60 V . Corriente max\+: ∼ 0,5 A)
\item Resistencia 33 Ω (De inserción. Tolerancia\+: ≤ 10 \%. Potencia\+: 1/4 − 1/2 W)
\end{DoxyItemize}

{\bfseries{Información de contacto\+:}}

{\bfseries{Hernán García Quijano\+:}} \href{mailto:hernan.garcia.quijano@alumnos.upm.es}{\texttt{ hernan.\+garcia.\+quijano@alumnos.\+upm.\+es}}

{\bfseries{Ángel Rodrigo Pérez Iglesias\+:}} \href{mailto:angelrodrigo.perez@alumnos.upm.es}{\texttt{ angelrodrigo.\+perez@alumnos.\+upm.\+es}} 